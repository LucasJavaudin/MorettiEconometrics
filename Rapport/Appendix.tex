\begin{appendices}
	\section{R codes}
	\begin{figure}[H]
		\caption{R code used to clean French data}
		\label{code_data_cleaning}
		\begin{lstlisting}
		###################
		#  Data Cleaning  #
		###################
		
		# In this part, we change the dataset to make it closer to the dataset of Moretti.
		
		# Remove the movies without any screen in France during the first week (667 movies).
		fr_df <- fr_df[!is.na(fr_df$seance_fr1),]
		# Remove the movies without any id_distributeur (4 movies).
		fr_df <- fr_df[!is.na(fr_df$id_distributeur),]
		
		# Set MoyennePresse and MoyenneSpectateur to the mean if no value is specified.
		mean_moy <- mean(fr_df[!is.na(fr_df$MoyennePresse), 'MoyennePresse'])
		fr_df[is.na(fr_df$MoyennePresse), 'MoyennePresse'] <- mean_moy
		mean_moy <- mean(fr_df[!is.na(fr_df$MoyenneSpectateur), 'MoyenneSpectateur'])
		fr_df[is.na(fr_df$MoyenneSpectateur), 'MoyenneSpectateur'] <- mean_moy
		
		# Repeat each columns 13 times.
		n <- nrow(fr_df)
		df <- fr_df[rep(1:n, each=13),]
		
		# Add a column to indicate the week.
		df$t <- rep(0:12, n)
		
		# Replace the variables for each week (e.g. 'entree_paris1') with a global variable (e.g. 'entree_paris')
		for (i in 0:12) {
		for (variable in c('entree_paris', 'seance_paris', 'entree_fr', 'seance_fr')) {
		# Concatenate the variable name with and indicator for the week (e.g. 'entree_paris1').
		variable_t <- paste(c(variable, toString(i+1)), collapse='')
		# For each week, the variable in the new df (e.g. 'entree_paris') is taken from the old df (e.g. 'entree_paris1').
		df[df$t==i, variable] <- fr_df[,variable_t]
		}
		}
		
		# Keep only the useful variables.
		df <- df[,c(1:6, 33:43, 70:85)]
		
		# Replace the NAs in seance_fr with zeros.
		df[is.na(df$seance_fr), 'seance_fr'] <- 0
		
		# Generate logarithm of sales and screens.
		df$log_entree_paris <- log(df$entree_paris + 1)
		df$log_seance_paris <- log(df$seance_paris + 1)
		df$log_entree_fr <- log(df$entree_fr + 1)
		df$log_seance_fr <- log(df$seance_fr + 1)
		
		# Variable id_distributeur is a factor.
		df$id_distributeur <- as.factor(df$id_distributeur)
		
		# Variable id is a factor (this is used for movie dummies with the package lfe).
		df$X <- as.factor(df$X)
		df$X.eff <- rnorm(nlevels(df$X))
		
		\end{lstlisting}
	\end{figure}
	\begin{figure}[H]
		\caption{R code used to obtain French surprises}
		\label{code_french_surprises}
		\begin{lstlisting}
		###############
		#  Surprises  #
		###############
		
		# In this part, we estimate the surprises of the movies.
		
		# Regression of first week sales on number of screens.
		regSurprise1 <- lm(log_entree_fr ~ log_seance_fr, data = df, subset = (t==0))
		# Including dummies for genre
		regSurprise2 <- lm(log_entree_fr ~ log_seance_fr + genre, data = df, subset = (t==0))
		# Including dummies for ratings 
		regSurprise3 <- lm(log_entree_fr ~ log_seance_fr + genre + interdiction, data = df, subset = (t==0))
		# Including dummies for distributor 
		regSurprise4 <- lm(log_entree_fr ~ log_seance_fr + genre + interdiction + id_distributeur, data = df, subset = (t==0))
		# Including dummies for month and week 
		regSurprise5 <- lm(log_entree_fr ~ log_seance_fr + genre + interdiction + id_distributeur + factor(mois) + factor(semaine), data = df, subset = (t==0))
		# Including dummies for year 
		regSurprise6 <- lm(log_entree_fr ~ log_seance_fr + genre + interdiction + id_distributeur + factor(mois) + factor(semaine) + factor(annee), data = df, subset = (t==0))
		# Including other variables
		regSurprise7 <- lm(log_entree_fr ~ log_seance_fr + genre + interdiction + id_distributeur + factor(mois) + factor(semaine) + factor(annee) + MoyennePresse + sigma_note_presse + PoidsCasting + pub, data = df, subset = (t==0))
		
		# Print a table with the results of the last regressions.
		stargazer(regSurprise1, regSurprise2, regSurprise3, regSurprise4, regSurprise5, regSurprise6, regSurprise7, type='text', keep=c('log_seance_fr'), omit.stat=c("f", "ser"), title='Regression of first-week entries on number of screens')
		
		# Surprises are defined as the residuals of the last regression.
		surprise <- residuals(regSurprise7)
		df$surprise <- rep(residuals(regSurprise3), each = 13)
		quantile(df$surprise, probs = c(0, .05, .1, .25, .5, .75, .9, .95, 1))
		
		# Generate additional variables for surprises.
		df$positive_surprise <- df$surprise >= 0
		q_surprise <- quantile(df$surprise, probs = c(1/3, 2/3))
		df$bottom_surprise <- df$surprise < q_surprise[1]
		df$middle_surprise <- df$surprise >= q_surprise[1] & df$surprise < q_surprise[2]
		df$top_surprise <- df$surprise >= q_surprise[2]
		
		\end{lstlisting}
	\end{figure}
	
	\begin{figure}[H]
		\caption{R code used to obtain French sales dynamics}
		\label{code_}
		\begin{lstlisting}
		###############################################
		#  Prediction 1: Surprises and Sale Dynamics  #
		###############################################
		
		# In this part, we study the difference in rate of decline between movies with a positive surprise and movies with a negative surprise.
		
		# Regression of sales on the interaction between time and surprises.
		# We use the command felm of the package lfe to compute linear regressions with thousands of dummies.
		regSaleDynamics1 <- felm(log_entree_fr ~ t | X, data = df)
		regSaleDynamics2 <- felm(log_entree_fr ~ t + t : surprise | X, data = df)
		regSaleDynamics3 <- felm(log_entree_fr ~ t + t : positive_surprise | X, data = df)
		regSaleDynamics4 <- felm(log_entree_fr ~ I(t * top_surprise)+ I(t * middle_surprise) + I(t * bottom_surprise) | X, data = df)
		
		# Print a table with the results of the regressions.
		stargazer(regSaleDynamics1, regSaleDynamics2, regSaleDynamics3, regSaleDynamics4, omit.stat=c("f", "ser"), title='Decline in box-office sales by opening week surprise')
		
		
		\end{lstlisting}
		
		
	\end{figure}
	
\end{appendices}
