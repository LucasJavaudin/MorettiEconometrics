\subsection{Size of the Social Network}\label{subsec2.4}

Consumers with a larger social network receive more feedbacks from their peers and thus they are able to evaluate more precisely the quality of the movie.
Hence, social learning should be stronger for consumers with a larger social network.
More formally, this prediction can be tested by estimating the models of the form:
\begin{equation}
	\ln (y_{jt})  = \beta_0 + \beta_1 t + \beta_2 (t \times S_j) + \beta_3 (t \times NS_j) + \beta_4 (t \times S_j \times NS_j) + d_j + u_{jt}
	\label{eqpred3}
\end{equation}
where $S_j$ is a dummy for positive surprise and $NS_j$ is a variable representing the network size of the movie $j$'s audience.
If social learning is stronger with higher values of $NS_j$, then the coefficient $\beta_4$ of the triple interaction between the time trend, the surprise and the network size should be positive.

In his article, Moretti uses two different measurement of network size.
First, he makes the assumption that teenagers have a more developed social network than adults and he estimates the model of equation \ref{eqpred3} with a dummy for teen movies.
He finds that the estimate of the coefficient $\beta_4$ is indeed positive.
However, the estimate is very weakly significant.
Additionally, there is no indicator for teen movies in the data and the way Moretti build a dummy for teen movies is quite surprising.
He uses \textit{genre1} (one of the three variables indicating the genre of the movies) and he considers that teen movies are movies of the genre action, adventure, comedy, fantasy, horror, sci-fi and suspense.
We would have appreciated more justification for the assumption that teenagers have a larger social network and for the way teen movies are defined.
To investigate further these issues, we used the two other variables indicating the genre of the movies: \textit{genre2} and \textit{genre3}.
Both variables have a category \textit{Children} and a category \textit{Youth}.
We used these two categories to define new dummies for teen movies.
Using \textit{genre2}, we find that the estimate of $\beta_4$ is significantly negative.
Using \textit{genre3}, we find that the estimate of $\beta_4$ is significantly positive.
We conclude that using teen movies is not a good way to test this prediction.

A second measurement of the size of the social network used by Moretti is the number of theaters broadcasting the movie during the opening week.
If a movie opens in lots of theater, the consumers should receive more feedbacks from their peers.
As expected, he estimates that the coefficient of $\beta_4$ is significantly positive.
We estimated the same model with the French data.
The results are reported in column (2) of table \ref{precisionsignal}.
We find that the coefficient of the triple interaction is indeed significantly positive.

In the French data, the variable \textit{tout public} is a dummy indicating movies which are suitable for any kind of audience.
We can assume that consumers have more feedbacks from their peers for movies opened to anyone.
Hence, we estimated the model of equation \ref{eqpred3} using the \textit{tout public} dummy to measure network size.
The results of the estimated are reported in column (1) of table \ref{precisionsignal}.
Consistently with our assumption, the coefficient of the triple interaction is significantly positive.

% Table created by stargazer v.5.2 by Marek Hlavac, Harvard University. E-mail: hlavac at fas.harvard.edu
% Date and time: Fri, Mar 02, 2018 - 05:12:24 PM
\begin{table}[!htbp] \centering 
  \caption{Precision of peers' signal} 
  \label{precisionsignal} 
\begin{tabular}{@{\extracolsep{5pt}}lcc} 
\\[-1.8ex]\hline 
\hline \\[-1.8ex] 
 & \multicolumn{2}{c}{\textit{Dependent variable:}} \\ 
\cline{2-3} 
\\[-1.8ex] & \multicolumn{2}{c}{log\_entree\_fr} \\ 
\\[-1.8ex] & (1) & (2)\\ 
\hline \\[-1.8ex] 
 $t$ & $-$0.663$^{***}$ & $-$0.451$^{***}$ \\ 
  & (0.007) & (0.005) \\ 
  & & \\ 
 $t$$\times$positive\_surprise & 0.061$^{***}$ & 0.076$^{***}$ \\ 
  & (0.010) & (0.006) \\ 
  & & \\ 
 $t$$\times$tout\_public & 0.115$^{***}$ &  \\ 
  & (0.008) &  \\ 
  & & \\ 
 $t$$\times$positive\_surprise$\times$tout\_public & 0.031$^{***}$ &  \\ 
  & (0.011) &  \\ 
  & & \\ 
 $t$$\times$seance\_fr\_first\_week &  & $-$0.033$^{***}$ \\ 
  &  & (0.001) \\ 
  & & \\ 
 $t$$\times$positive\_surprise$\times$seance\_fr\_first\_week &  & 0.011$^{***}$ \\ 
  &  & (0.001) \\ 
  & & \\ 
\hline \\[-1.8ex] 
Observations & 26,598 & 26,598 \\ 
R$^{2}$ & 0.856 & 0.867 \\ 
Adjusted R$^{2}$ & 0.844 & 0.856 \\ 
\hline 
\hline \\[-1.8ex] 
\textit{Note:}  & \multicolumn{2}{r}{$^{*}$p$<$0.1; $^{**}$p$<$0.05; $^{***}$p$<$0.01} \\ 
\end{tabular} 
\end{table}

\subsection{Does learning decline over time?}\label{subsec2.5}

The model predicts that the effects of positive and negative surprises should decline over time.
More precisely, sales profile should be a concave function of time for positive surprises and a convex function of time for negative surprises.
To test this prediction, we need to estimate the sales profile which is assumed to be a quadratic function of time.
Therefore, we estimate the following model:
\begin{equation*}
	\ln (y_{jt}) = \beta_0 + \beta_1 t + \beta_2 t^2 + \beta_3 (t \times S_j) + \beta_4 (t^2 \times S_j) + d_j + u_{jt}
\end{equation*}
where $S_j$ is a dummy for positive surprise.
The results are reported in table \ref{pred4}.
The second derivative of log of entries for negative-surprise movies is 
\begin{equation*}
	\left.\frac{\partial^2 y_{jt}}{\partial t^2}\right|_{S_j=0}} = 2 \beta_2.
\end{equation*}
The second derivative of log of entries for positive-surprise movies is 
\begin{equation*}
	\left.\frac{\partial^2 y_{jt}}{\partial t^2}\right|_{S_j=1}} = 2 (\beta_2 + \beta_4).
\end{equation*}
We can test the hypothesis of convexity ($2\beta_2 > 0$) and the hypothesis of concavity ($2(\beta_2 + \beta_4) < 0$) with Student tests.
For instance, to test $H_0:~2(\beta_2 + \beta_4) < 0$ against $H_1:~2(\beta_2 + \beta_4) > 0$, the $t$ statistic is
\begin{equation*}
	t = \frac{2(\hat{\beta}_2 + \hat{\beta}_4)}{\text{se}(2(\hat{\beta}_2 + \hat{\beta}_4))} 
\end{equation*}
where $\text{se}(2(\hat{\beta}_2 + \hat{\beta}_4)) = 2[{\text{Var}(\hat{\beta}_2) + \text{Var}(\hat{\beta}_4) + 2\cdot \text{Cov}(\hat{\beta}_2, \hat{\beta}_4)]^{1/2}$ is the standard error of $2(\hat{\beta}_2 + \hat{\beta}_4)$.

With the US data, both hypotheses cannot be rejected with a good confidence.
With French data, the $p$-value for the test of convexity of negative-surprise movies is really close to 0 ($t \approx 36.72$ and $p \approx 0$). 
However, the hypothesis of concavity of positive-surprise movies must be rejected ($t \approx 9.41$ and $p \approx 1$).
What we can say however is that the sales profile of positive-surprise movies is \textit{more concave} than the sales profile of negative-surprise movies because the estimates show that the coefficient $\beta_4$ is significantly negative.
These statements are confirmed by the graphs of the sales profile of figure \ref{part2.1_plot_moretti} where the sales profile of negative-surprise movies is clearly convex and the sales profile of positive-surprise movies seems linear.

% Table created by stargazer v.5.2 by Marek Hlavac, Harvard University. E-mail: hlavac at fas.harvard.edu
% Date and time: Fri, Mar 02, 2018 - 05:12:25 PM
\begin{table}[!htbp] \centering 
  \caption{Convexity of the sales profile} 
  \label{pred4} 
\begin{tabular}{@{\extracolsep{5pt}}lc} 
\\[-1.8ex]\hline 
\hline \\[-1.8ex] 
 & \multicolumn{1}{c}{\textit{Dependent variable:}} \\ 
\cline{2-2} 
\\[-1.8ex] & log\_entree\_fr \\ 
\hline \\[-1.8ex] 
 $t$ & $-$0.978$^{***}$ \\ 
  & (0.011) \\ 
  & \\ 
 $t^2$ & 0.034$^{***}$ \\ 
  & (0.001) \\ 
  & \\ 
 $t$$\times$positive\_surprise & 0.393$^{***}$ \\ 
  & (0.016) \\ 
  & \\ 
 $t^2$$\times$positive\_surprise & $-$0.026$^{***}$ \\ 
  & (0.001) \\ 
  & \\ 
\hline \\[-1.8ex] 
Observations & 26,598 \\ 
R$^{2}$ & 0.861 \\ 
Adjusted R$^{2}$ & 0.850 \\ 
\hline 
\hline \\[-1.8ex] 
\textit{Note:}  & \multicolumn{1}{r}{$^{*}$p$<$0.1; $^{**}$p$<$0.05; $^{***}$p$<$0.01} \\ 
\end{tabular} 
\end{table}
